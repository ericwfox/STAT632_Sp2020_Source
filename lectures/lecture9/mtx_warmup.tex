\documentclass[11pt, fleqn]{article}\usepackage[]{graphicx}\usepackage[]{color}
%% maxwidth is the original width if it is less than linewidth
%% otherwise use linewidth (to make sure the graphics do not exceed the margin)
\makeatletter
\def\maxwidth{ %
  \ifdim\Gin@nat@width>\linewidth
    \linewidth
  \else
    \Gin@nat@width
  \fi
}
\makeatother

\definecolor{fgcolor}{rgb}{0.345, 0.345, 0.345}
\newcommand{\hlnum}[1]{\textcolor[rgb]{0.686,0.059,0.569}{#1}}%
\newcommand{\hlstr}[1]{\textcolor[rgb]{0.192,0.494,0.8}{#1}}%
\newcommand{\hlcom}[1]{\textcolor[rgb]{0.678,0.584,0.686}{\textit{#1}}}%
\newcommand{\hlopt}[1]{\textcolor[rgb]{0,0,0}{#1}}%
\newcommand{\hlstd}[1]{\textcolor[rgb]{0.345,0.345,0.345}{#1}}%
\newcommand{\hlkwa}[1]{\textcolor[rgb]{0.161,0.373,0.58}{\textbf{#1}}}%
\newcommand{\hlkwb}[1]{\textcolor[rgb]{0.69,0.353,0.396}{#1}}%
\newcommand{\hlkwc}[1]{\textcolor[rgb]{0.333,0.667,0.333}{#1}}%
\newcommand{\hlkwd}[1]{\textcolor[rgb]{0.737,0.353,0.396}{\textbf{#1}}}%
\let\hlipl\hlkwb

\usepackage{framed}
\makeatletter
\newenvironment{kframe}{%
 \def\at@end@of@kframe{}%
 \ifinner\ifhmode%
  \def\at@end@of@kframe{\end{minipage}}%
  \begin{minipage}{\columnwidth}%
 \fi\fi%
 \def\FrameCommand##1{\hskip\@totalleftmargin \hskip-\fboxsep
 \colorbox{shadecolor}{##1}\hskip-\fboxsep
     % There is no \\@totalrightmargin, so:
     \hskip-\linewidth \hskip-\@totalleftmargin \hskip\columnwidth}%
 \MakeFramed {\advance\hsize-\width
   \@totalleftmargin\z@ \linewidth\hsize
   \@setminipage}}%
 {\par\unskip\endMakeFramed%
 \at@end@of@kframe}
\makeatother

\definecolor{shadecolor}{rgb}{.97, .97, .97}
\definecolor{messagecolor}{rgb}{0, 0, 0}
\definecolor{warningcolor}{rgb}{1, 0, 1}
\definecolor{errorcolor}{rgb}{1, 0, 0}
\newenvironment{knitrout}{}{} % an empty environment to be redefined in TeX

\usepackage{alltt}
\usepackage{amsmath}
\usepackage{amssymb}
\usepackage{geometry}
\usepackage{graphicx}
\usepackage{bm}
\usepackage{url}
\usepackage{hyperref}
\usepackage{enumerate}
\usepackage{fullpage}
\IfFileExists{upquote.sty}{\usepackage{upquote}}{}
\begin{document}
\setlength\parindent{0pt}

\large \textbf{Matrix Warmup}\footnote{This is based on a worksheet created by Professor Charlotte Wickham of Oregon State University.}\\

\normalsize Let
\[
\bm{A} =
\begin{pmatrix}
1 & 1\\
1 & 2\\
1 & 3 
\end{pmatrix},
\quad
\bm{B} = 
\begin{pmatrix}
4 & 7\\
1 & 2
\end{pmatrix},
\quad
\bm{x} = 
\begin{pmatrix}
4\\
2
\end{pmatrix},
\quad
c = 3,
\quad
\bm{I}_3 = 
\begin{pmatrix}
1 & 0 & 0\\
0 & 1 & 0\\
0 & 0 & 1
\end{pmatrix}
\]

Compute each of the following, or specify that it's not possible.
\begin{enumerate}
\item $\bm{A} + \bm{A}$\\
\vspace{11pt}

\item $\bm{x} + \bm{A}$\\
\vspace{11pt}

\item $c \bm{A}$\\
\vspace{11pt}

\item $\bm{A} \bm{x}$\\
\vspace{11pt}

\item $\bm{x} \bm{A}$\\
\vspace{11pt}

\item $\bm{I}_3 \bm{A}$\\
\vspace{11pt}

\item $\bm{A'}$\\
\vspace{11pt}

\item $\det(\bm{B})$\\
\vspace{11pt}

\item $\bm{B}^{-1}$\\
\vspace{11pt}

\item $\bm{x'}\bm{x}$\\
\end{enumerate}
\clearpage

To check your answers, you can use R:
\begin{knitrout}
\definecolor{shadecolor}{rgb}{0.969, 0.969, 0.969}\color{fgcolor}\begin{kframe}
\begin{alltt}
\hlstd{A} \hlkwb{<-} \hlkwd{matrix}\hlstd{(}\hlkwd{c}\hlstd{(}
    \hlnum{1}\hlstd{,} \hlnum{1}\hlstd{,}
    \hlnum{1}\hlstd{,} \hlnum{2}\hlstd{,}
    \hlnum{1}\hlstd{,} \hlnum{3}
  \hlstd{),}
  \hlkwc{ncol}\hlstd{=}\hlnum{2}\hlstd{,} \hlkwc{byrow}\hlstd{=T)}
\hlstd{B} \hlkwb{<-} \hlkwd{matrix}\hlstd{(}\hlkwd{c}\hlstd{(}
    \hlnum{4}\hlstd{,} \hlnum{7}\hlstd{,}
    \hlnum{1}\hlstd{,} \hlnum{2}
  \hlstd{),}
  \hlkwc{ncol}\hlstd{=}\hlnum{2}\hlstd{,} \hlkwc{byrow}\hlstd{=T)}

\hlstd{x} \hlkwb{<-} \hlkwd{matrix}\hlstd{(}\hlkwd{c}\hlstd{(}\hlnum{4}\hlstd{,} \hlnum{2}\hlstd{),} \hlkwc{ncol}\hlstd{=}\hlnum{1}\hlstd{,} \hlkwc{byrow}\hlstd{=T)}

\hlstd{c} \hlkwb{<-} \hlnum{3}

\hlstd{I_3} \hlkwb{<-} \hlkwd{diag}\hlstd{(}\hlkwc{nrow} \hlstd{=} \hlnum{3}\hlstd{)}
\end{alltt}
\end{kframe}
\end{knitrout}

\begin{knitrout}
\definecolor{shadecolor}{rgb}{0.969, 0.969, 0.969}\color{fgcolor}\begin{kframe}
\begin{alltt}
\hlcom{# 1}
\hlstd{A} \hlopt{+} \hlstd{A}
\hlcom{# 2}
\hlstd{x} \hlopt{+} \hlstd{A}
\hlcom{# 3}
\hlstd{c} \hlopt{*} \hlstd{A}
\hlcom{# 4}
\hlstd{A} \hlopt \hlstd{x}
\hlcom{# 5}
\hlstd{x} \hlopt \hlstd{A}
\hlcom{# 6}
\hlstd{I_3} \hlopt \hlstd{A}
\hlcom{# 7}
\hlkwd{t}\hlstd{(A)}
\hlcom{# 8}
\hlkwd{det}\hlstd{(B)}
\hlcom{# 9}
\hlkwd{solve}\hlstd{(B)}
\hlcom{# 10}
\hlkwd{t}\hlstd{(x)} \hlopt \hlstd{x}
\end{alltt}
\end{kframe}
\end{knitrout}


A nice reference about matrix algebra in R:\\
\url{https://www.statmethods.net/advstats/matrix.html}\\

\end{document}
