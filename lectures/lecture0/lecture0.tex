\documentclass[10pt]{beamer}
\usepackage{amsmath}
\usepackage{amssymb}
\usepackage{geometry}
\usepackage{graphicx}
\usepackage{url}

\begin{document}

\begin{frame}
\large
Lecture 0\\
Introduction and Syllabus\\
STAT 632, Spring 2020
\end{frame}

\begin{frame}{Course Topics}
\textbf{Linear Regression}: modeling the relationship between a single continuous response variable $Y$, and one or more explanatory (or predictor) variables $X_1, X_2, \cdots, X_p$.
\vspace{10pt}

\textbf{Logistic Regression}: modeling the relationship between a single binary response variable $Y$, which takes on values 0 or 1, and one or more explanatory variables $X_1, X_2, \cdots, X_p$.\\
\vspace{10pt}

The course will cover practical applications to a variety of real data sets, the mathematical theory of the linear and logistic regression model, and computation in R.
\end{frame}

\begin{frame}{Motivation}
Regression modeling has several objectives:
\vspace{5pt}
\begin{itemize}
\item Making predictions for future or unknown values of the response variable, and evaluating the uncertainty in those predictions.
\vspace{5pt}
\item Assessing the relationship between the response and explanatory variables.
\vspace{5pt}
\item Providing insight into the data structure (e.g., checking for unusual or influential observations)
\end{itemize}
\end{frame}

\begin{frame}{Additional Topics}
We may also cover several modern statistical modeling techniques related to linear regression:
\vspace{5pt}
\begin{itemize}
\item Regularization methods such as ridge regression and LASSO that are useful when there are many predictor variables ($p \approx n$ or $p > n$).
\vspace{5pt}
\item Decision trees and random forest models that are useful when there are many predictor variables with nonlinear relationships.
\vspace{5pt}
\item Generalized least squares estimation, which is a method that can be used when the data are autocorrelated (e.g., time series data)
\end{itemize}
\end{frame}

\begin{frame}{Grading}
\begin{itemize}
\item 40\% Two Midterm Exams (20\% each)
\item 30\% Homework
\item 20\% Project Paper 
\item 10\% Presentation
\end{itemize}
\end{frame}

\begin{frame}
\textbf{Exams}: There will be two midterm exams, each worth 20\% of your grade.  There will be no final exam.\\
%Mention that one exam might be closed-book (allowed 2 pages of notes)
\vspace{10pt}
\textbf{Homework}: There will be biweekly homework assignments.  You should receive full credit, or close to full credit, if you put in a reasonable effort, and turn in your work on time.  Also, I may not grade every problem, but I will post solutions on Blackboard.  You are encouraged to use R Markdown for data analysis and coding exercises.  I will drop your lowest scoring homework assignment.\\
\vspace{10pt}
\textbf{Project and Presentation}:  For the project you will need to find a data set of interest, and then conduct a regression analysis using that data set.  You will be required to give a presentation on your project during the last two weeks of class.  For the final project you are also encouraged to use a modern method (e.g., random forests, LASSO).
\end{frame}

\begin{frame}{Textbooks}
Simon Sheather. \emph{A Modern Approach to Regression with R}, Springer, 2009.\\
\vspace{4pt}
Free electronic version: \url{http://library.csueastbay.edu/home}\\
Data sets and R code: \url{http://gattonweb.uky.edu/sheather/book/}\\
\vspace{4pt}
This will be the main textbook for the course.  If you wish, you may buy a hard copy through the CSUEB student store, Amazon, or Springer.
\vspace{15pt}

James, G., Witten, D., Hastie, T., and Tibshirani, R. \emph{An Introduction to Statistical Learning with Applications in R}. Springer, 2013.\\
\vspace{4pt}
Free PDF version: \url{http://www-bcf.usc.edu/~gareth/ISL/}\\
\vspace{4pt}
We will reference this textbook when covering statistical learning topics such as cross-validation, LASSO, and random forests.  
%The book also provides a very accessible introduction to linear and logistic regression.
\end{frame}

\begin{frame}{Software}
We will use R and RStudio for data analysis and statistical modeling.  This course assumes some familiarity with computer programming.  We will try to cover all of the following R topics:
\begin{itemize}
\item Data visualization and summary statistics (base R and \texttt{ggplot2})
\item Linear regression modeling with the \texttt{lm()} function
\item Logistic regression modeling with the \texttt{glm()} function
\item Random forest modeling with the \texttt{randomForest} package
\item LASSO and ridge regression with the \texttt{glmnet} package
\item Report writing and reproducible research (R Markdown, knitr)
\end{itemize}
\vspace{5pt}
\end{frame}

\begin{frame}{Software}
You are also encouraged to learn LaTeX for mathematical typesetting.  LaTeX can be combined with R using Markdown or knitr.\\
\begin{itemize}
\item Download LaTeX: \url{https://www.latex-project.org/}
\item Resource for learning LaTeX: \url{https://www.overleaf.com/learn/latex/Main_Page}
\end{itemize}
\end{frame}

\end{document}